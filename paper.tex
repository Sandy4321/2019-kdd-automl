\documentclass{article}

\usepackage{microtype}
\usepackage{graphicx}
\usepackage{subfigure}
\usepackage{booktabs}
\usepackage{hyperref}
\usepackage{listings}

\lstset{
  aboveskip=12pt,
  belowskip=0pt,
  basicstyle=\fontsize{8.5pt}{8.5pt}\ttfamily,
  breaklines=true,
  captionpos=b,
  frame=tb,
  framexbottommargin=5pt,
  framextopmargin=5pt,
}

\usepackage{enumitem}

\setlist[description]{
  itemsep=0pt,
  topsep=0pt,
  labelindent=1em,
  leftmargin=1em,
}

\setlist[enumerate]{
  itemsep=0pt,
  topsep=0pt,
}

\setlist[itemize]{
  itemsep=0pt,
  topsep=0pt,
  labelindent=1em,
  leftmargin=1em,
}

\usepackage[accepted]{sysml2019}

\begin{document}

\twocolumn[
\sysmltitle{Packaging Machine Learning Models for AutoML with Guild AI}

\begin{sysmlauthorlist}
\sysmlauthor{Garrett Smith}{guild}
\end{sysmlauthorlist}

\sysmlaffiliation{guild}{Guild AI, Chicago, Illinois, USA}

\sysmlcorrespondingauthor{Garrett Smith}{garrett@guild.ai}

\vskip 0.3in

\begin{abstract}

  Guild AI is an open source toolkit used to run, track, and compare
  machine learning experiments using hyperparameter tuning and
  architecture search. Guild makes this facility available through a
  novel approach to software packaging and distribution where models
  are installed, trained, and optimized as black boxes. In this
  extended paper abstract, we consider the use of \emph{packaged
    machine learning models} to facilitate AutoML. Specifically, we
  present Guild AI as a tool for publishing models to an open,
  federated ecosystem for use in training and automated optimization.

\end{abstract}
]

\printAffiliationsAndNotice{}

\section{Guild AI}

\textbf{Experiments.} Guild AI runs experiment trials as operating
system processes and tracks results in separate run directories. Each
run directory contains artifacts created during the trial along with
metadata such as model information, start and stop times, logs, and
run status. Run artifacts commonly include training checkpoints,
prepared data sets, generated content such as images and audio, logs,
and project source code snapshots.

\textbf{Hyperparameter Search.} Guild AI supports a variety of
hyperparameter search methods including grid search, random search,
and Bayesian optimization. Guild's facility for optimization is
customizable, letting model authors implement their own
optimizers. Sequential optimizers make use of experiment results to
propose new hyperparameter values to minimize an objective function.

Guild does not provide software libraries to drive
experiments. Instead it evaluates experiment artifacts (logs, output,
etc.) to read scalar output for use in hyperparameter proposals. This
approach lets Guild work with unmodified training scripts. Models
never intended for hyperparameter optimization may be optimized using
this approach.

\textbf{Architecture Search.} Guild AI supports architecture search in
two ways. First, Guild selects from a set of related model
architectures for search candidates. This may include any of Guild's
supported search methods: grid, random, and Bayesian
optimization. Second, Guild drives neural architecture search (NAS)
algorithms using historical training data.

\textbf{Models and Operations.} Guild AI lets authors define models
and related operations in a \emph{Guild file}. A Guild file is named
\texttt{guild.yml} and located in the project root directory. A
\emph{model} represents a machine learning model. Models define
\emph{operations}, which are run using Guild's \texttt{run}
command. Listing~\ref{lst:guild-file} illustrates a simple Guild file
that defines a model \texttt{resnet50} with \texttt{train} and
\texttt{evaluate} operations.

\begin{figure}
\begin{lstlisting}[
    caption=A simple Guild file,
    label={lst:guild-file}]
model: resnet50
description:
  ResNet-50 image classifier
operations:
  train:
    main: train_resnet50
    flags:
      batch-size: 32
      train-steps: 1000
  evaluate:
    main: eval_resnet50
\end{lstlisting}
\end{figure}

Guild AI does not prescribe a particular set of operations for a
model. Model developers are free to name operations as they see fit,
though Guild AI does encourage the use of consistent naming
conventions.

Below are some common operation names.

\begin{description}
\item[prepare] prepare a data set for training
\item[train] train a model from scratch
\item[transfer-learn] train a model using transfer learning
\item[finetune] finetune a trained model
\item[evaluate] evaluate/test a model
\item[classify] use a trained model for classification
\end{description}

\textbf{Reproducibility.} Guild files enable model \emph{reuse} and
\emph{reproducibility}. They let users to run experiments with a
single command (e.g. \texttt{guild run resnet50:train}), capture the
results as a unique experiment (see \emph{Experiments} above), and
compare the results to previous experiments.

\textbf{Packages.} Projects containing a Guild file, along with
supporting source code, may be packaged into a redistributable
artifact, or a \emph{package}. Guild packages are standard Python
wheel distributions \cite{python-dist}.

Packages can be installed using pip \cite{pip} or Guild's
\texttt{install} command. Once installed, a user has access to the
packaged model operations without the need to access source
code. Packages allow rapid black box model experimentation.

\textbf{Related Work and Differences}. There are several open source
projects that address similar goals in machine learning. Prominent
projects include:

\begin{itemize}[noitemsep, topsep=0pt, partopsep=0pt]
\item ModelDB \cite{modeldb}
\item MLFlow \cite{mlflow}
\item Polyaxon \cite{polyaxon}
\item datmo \cite{datmo}
\end{itemize}

Automation tools in machine learning commonly require tool-specific
changes to project source code. This generally includes adoption of a
Python API for controlling experiments, accessing hyperparameters and
logging results. Automation tools may further require additional
system software such as databases, container management systems, and
job schedulers.

Each requirement that a tool imposes on a user presents a barrier to
the goal of systematic machine learning. In some cases, users may view
the cost of supporting reproducibility to outweigh the benefits.

To encourage systematic machine learning, Guild runs unmodified
learning scripts without requiring the installation and maintenance of
external systems (e.g. databases, containers, orchestration services,
etc.) Once Guild is installed, for example, an unmodified Python
script \texttt{train.py} that supports hyperparameters
\texttt{n\_layers} and \texttt{lr} can be run using:

{\footnotesize
\begin{verbatim}
$ guild run train.py n_layers=2 lr=1e-4
\end{verbatim}}

With this command, the user generates a unique experiment with the
specified hyperparameters, which can be analyzed, compared to other
trials, and shared.

Guild additionally supports easy packaging, distribution, and
installation of research work in the support of machine learning
reproducibility. To our knowledge, no other tool provides this
feature.

\section{Packaging and AutoML}

AutoML is a broad term encompassing forms of automated machine
learning. In this section, we consider Guild AI packages in this
context.

Guild packages support fast installation and reuse of machine learning
models. Consider this sequence of commands:

{\footnotesize
\begin{verbatim}
 $ guild install gpkg.resnet

 $ guild run gpkg.resnet/resnet50:train \
   images=~/dogs-and-cats
\end{verbatim}}

The \texttt{install} command instructs Guild to download and install
the package \texttt{gpkg.resnet}. This package was created and
published earlier by a package author. The package contains several
ResNet image classifier architectures (models), each supporting a
\texttt{train} operation.

The \texttt{run} command runs the \texttt{train} operation for the
\texttt{resnet50} model, which is provided by the installed
package. This operation trains an image classifier on the specified
data set.

Next, consider the command:

{\footnotesize
\begin{verbatim}
 $ guild run gpkg.resnet/resnet50:train
    images=~/dogs-and-cats
    --optimizer gaussian
\end{verbatim}}

This command runs the \texttt{train} operation, but this time with a
Bayesian \texttt{optimizer} using Gaussian processes. This single
commands generates a series of trials that attempt to minimize
validation loss after training. This illustrates Guild's built-in
hyperparameter optimization support.

Finally, consider the command:

{\footnotesize
\begin{verbatim}
 $ guild run gpkg.resnet{34,50,101,152}:train
    images=~/dogs-and-cats
    --optimizer gaussian
\end{verbatim}}

This commands instructs Guild to run Bayesian optimization over four
different model architectures: ResNet-34, ResNet-50, ResNet-101, and
ResNet-152. This is a hybrid search, using grid search over a discrete
set of architectures and hyperparamter tuning within each
architecture.

By supporting packaged models, Guild lets users quickly experiment
over multiple model architectures using a variety of search
methods. Each trial is tracked as a separate experiment regardless of
how it was started (manually or via AutoML).

\section{Open, Federated Ecosystem}

Guild packages are deployed to PyPI \cite{PyPI}, the standard
repository for Python packages. Model authors are free to publish
machine learning packages to PyPI in the same way software developers
are free to publish software packages.

There is no single controller or arbiter for package publishing. All
model publishers are supported provided they abide by PyPI terms of
service.

Similarly, users are free to install only the packages they're
interested in using. There is no controlling agent that determines
what models may be installed. As long as a published model provides
the expected operations (e.g. \emph{train}) it can participate in
AutoML trials like any other.

Provided there is a sufficiently diverse set of available models for a
given task, this ecosystem supports automated learning over a nearly
exhaustive range of architectures and hyperparameters.

\section{Summary}

We consider Guild AI as a tool for AutoML. Specifically, we examine
Guild's support for packaging machine learning models and its support
for automating experiments, hyperparameter tuning, and architecture
search.

By standardizing models and operations, Guild lets users reuse and
recreate experiments by running simple operations. In turn these
operations can be automated to optimize objectives across different
models.

By supporting packages on PyPI, Guild supports an open, federated
ecosystem of machine learning packages that can evolve with new
models. This scheme facilitates open exchange, centered on concrete
experimentation where users measure and compare trial results.

\bibliography{paper}
\bibliographystyle{sysml2019}

\end{document}
